% Options for packages loaded elsewhere
\PassOptionsToPackage{unicode}{hyperref}
\PassOptionsToPackage{hyphens}{url}
\documentclass[
]{article}
\usepackage{xcolor}
\usepackage[margin=1in]{geometry}
\usepackage{amsmath,amssymb}
\setcounter{secnumdepth}{-\maxdimen} % remove section numbering
\usepackage{iftex}
\ifPDFTeX
  \usepackage[T1]{fontenc}
  \usepackage[utf8]{inputenc}
  \usepackage{textcomp} % provide euro and other symbols
\else % if luatex or xetex
  \usepackage{unicode-math} % this also loads fontspec
  \defaultfontfeatures{Scale=MatchLowercase}
  \defaultfontfeatures[\rmfamily]{Ligatures=TeX,Scale=1}
\fi
\usepackage{lmodern}
\ifPDFTeX\else
  % xetex/luatex font selection
\fi
% Use upquote if available, for straight quotes in verbatim environments
\IfFileExists{upquote.sty}{\usepackage{upquote}}{}
\IfFileExists{microtype.sty}{% use microtype if available
  \usepackage[]{microtype}
  \UseMicrotypeSet[protrusion]{basicmath} % disable protrusion for tt fonts
}{}
\makeatletter
\@ifundefined{KOMAClassName}{% if non-KOMA class
  \IfFileExists{parskip.sty}{%
    \usepackage{parskip}
  }{% else
    \setlength{\parindent}{0pt}
    \setlength{\parskip}{6pt plus 2pt minus 1pt}}
}{% if KOMA class
  \KOMAoptions{parskip=half}}
\makeatother
\usepackage{color}
\usepackage{fancyvrb}
\newcommand{\VerbBar}{|}
\newcommand{\VERB}{\Verb[commandchars=\\\{\}]}
\DefineVerbatimEnvironment{Highlighting}{Verbatim}{commandchars=\\\{\}}
% Add ',fontsize=\small' for more characters per line
\usepackage{framed}
\definecolor{shadecolor}{RGB}{248,248,248}
\newenvironment{Shaded}{\begin{snugshade}}{\end{snugshade}}
\newcommand{\AlertTok}[1]{\textcolor[rgb]{0.94,0.16,0.16}{#1}}
\newcommand{\AnnotationTok}[1]{\textcolor[rgb]{0.56,0.35,0.01}{\textbf{\textit{#1}}}}
\newcommand{\AttributeTok}[1]{\textcolor[rgb]{0.13,0.29,0.53}{#1}}
\newcommand{\BaseNTok}[1]{\textcolor[rgb]{0.00,0.00,0.81}{#1}}
\newcommand{\BuiltInTok}[1]{#1}
\newcommand{\CharTok}[1]{\textcolor[rgb]{0.31,0.60,0.02}{#1}}
\newcommand{\CommentTok}[1]{\textcolor[rgb]{0.56,0.35,0.01}{\textit{#1}}}
\newcommand{\CommentVarTok}[1]{\textcolor[rgb]{0.56,0.35,0.01}{\textbf{\textit{#1}}}}
\newcommand{\ConstantTok}[1]{\textcolor[rgb]{0.56,0.35,0.01}{#1}}
\newcommand{\ControlFlowTok}[1]{\textcolor[rgb]{0.13,0.29,0.53}{\textbf{#1}}}
\newcommand{\DataTypeTok}[1]{\textcolor[rgb]{0.13,0.29,0.53}{#1}}
\newcommand{\DecValTok}[1]{\textcolor[rgb]{0.00,0.00,0.81}{#1}}
\newcommand{\DocumentationTok}[1]{\textcolor[rgb]{0.56,0.35,0.01}{\textbf{\textit{#1}}}}
\newcommand{\ErrorTok}[1]{\textcolor[rgb]{0.64,0.00,0.00}{\textbf{#1}}}
\newcommand{\ExtensionTok}[1]{#1}
\newcommand{\FloatTok}[1]{\textcolor[rgb]{0.00,0.00,0.81}{#1}}
\newcommand{\FunctionTok}[1]{\textcolor[rgb]{0.13,0.29,0.53}{\textbf{#1}}}
\newcommand{\ImportTok}[1]{#1}
\newcommand{\InformationTok}[1]{\textcolor[rgb]{0.56,0.35,0.01}{\textbf{\textit{#1}}}}
\newcommand{\KeywordTok}[1]{\textcolor[rgb]{0.13,0.29,0.53}{\textbf{#1}}}
\newcommand{\NormalTok}[1]{#1}
\newcommand{\OperatorTok}[1]{\textcolor[rgb]{0.81,0.36,0.00}{\textbf{#1}}}
\newcommand{\OtherTok}[1]{\textcolor[rgb]{0.56,0.35,0.01}{#1}}
\newcommand{\PreprocessorTok}[1]{\textcolor[rgb]{0.56,0.35,0.01}{\textit{#1}}}
\newcommand{\RegionMarkerTok}[1]{#1}
\newcommand{\SpecialCharTok}[1]{\textcolor[rgb]{0.81,0.36,0.00}{\textbf{#1}}}
\newcommand{\SpecialStringTok}[1]{\textcolor[rgb]{0.31,0.60,0.02}{#1}}
\newcommand{\StringTok}[1]{\textcolor[rgb]{0.31,0.60,0.02}{#1}}
\newcommand{\VariableTok}[1]{\textcolor[rgb]{0.00,0.00,0.00}{#1}}
\newcommand{\VerbatimStringTok}[1]{\textcolor[rgb]{0.31,0.60,0.02}{#1}}
\newcommand{\WarningTok}[1]{\textcolor[rgb]{0.56,0.35,0.01}{\textbf{\textit{#1}}}}
\usepackage{graphicx}
\makeatletter
\newsavebox\pandoc@box
\newcommand*\pandocbounded[1]{% scales image to fit in text height/width
  \sbox\pandoc@box{#1}%
  \Gscale@div\@tempa{\textheight}{\dimexpr\ht\pandoc@box+\dp\pandoc@box\relax}%
  \Gscale@div\@tempb{\linewidth}{\wd\pandoc@box}%
  \ifdim\@tempb\p@<\@tempa\p@\let\@tempa\@tempb\fi% select the smaller of both
  \ifdim\@tempa\p@<\p@\scalebox{\@tempa}{\usebox\pandoc@box}%
  \else\usebox{\pandoc@box}%
  \fi%
}
% Set default figure placement to htbp
\def\fps@figure{htbp}
\makeatother
\setlength{\emergencystretch}{3em} % prevent overfull lines
\providecommand{\tightlist}{%
  \setlength{\itemsep}{0pt}\setlength{\parskip}{0pt}}
\usepackage{bookmark}
\IfFileExists{xurl.sty}{\usepackage{xurl}}{} % add URL line breaks if available
\urlstyle{same}
\hypersetup{
  hidelinks,
  pdfcreator={LaTeX via pandoc}}

\author{}
\date{\vspace{-2.5em}}

\begin{document}

\section{Daily footsteps activity
analysis}\label{daily-footsteps-activity-analysis}

This report analyzes daily step activity data from
\href{https://d396qusza40orc.cloudfront.net/repdata\%2Fdata\%2Factivity.zip}{Activity\_monitoring\_dataset}

The analysis explores patterns in daily movement, missing data handling,
and differences between weekday and weekend activity.\\
I will use R for data manipulation and visualization, as the main goal
of this analysis is to become familiar with working in an \textbf{R
Markdown} file.

\subsection{Loading the data}\label{loading-the-data}

The variables included in this dataset are:

\begin{itemize}
\tightlist
\item
  \textbf{steps}: Number of steps taking in a 5-minute interval (missing
  values are coded as NA)
\item
  \textbf{date}: The date on which the measurement was taken in
  YYYY-MM-DD format
\item
  \textbf{interval}: Identifier for the 5-minute interval in which
  measurement was taken
\end{itemize}

\begin{Shaded}
\begin{Highlighting}[]
\FunctionTok{library}\NormalTok{(dplyr)}
\end{Highlighting}
\end{Shaded}

\begin{Shaded}
\begin{Highlighting}[]
\NormalTok{activity}\OtherTok{\textless{}{-}}\FunctionTok{read.csv}\NormalTok{(}\StringTok{"activity.csv"}\NormalTok{)}
\end{Highlighting}
\end{Shaded}

\subsection{What is the mean of the total number of steps taken per
day?}\label{what-is-the-mean-of-the-total-number-of-steps-taken-per-day}

I calculate the total daily steps by summing the data by date and
visualize their distribution with a histogram to show daily activity
patterns.

\begin{Shaded}
\begin{Highlighting}[]
\NormalTok{total\_per\_day}\OtherTok{\textless{}{-}}\NormalTok{ activity }\SpecialCharTok{\%\textgreater{}\%}
    \FunctionTok{group\_by}\NormalTok{(date) }\SpecialCharTok{\%\textgreater{}\%}
\FunctionTok{summarise}\NormalTok{(}\AttributeTok{total\_steps=}\FunctionTok{sum}\NormalTok{(steps,}\AttributeTok{na.rm =} \ConstantTok{TRUE}\NormalTok{))}
\FunctionTok{hist}\NormalTok{(total\_per\_day}\SpecialCharTok{$}\NormalTok{total\_steps,}
     \AttributeTok{main =} \StringTok{"Total Steps per Day"}\NormalTok{,}
     \AttributeTok{xlab =} \StringTok{"Total Steps"}\NormalTok{,}
     \AttributeTok{ylab =} \StringTok{"Number of Days"}\NormalTok{,}
     \AttributeTok{col =} \StringTok{"blue"}\NormalTok{,}
     \AttributeTok{breaks =} \DecValTok{20}\NormalTok{)}
\end{Highlighting}
\end{Shaded}

\pandocbounded{\includegraphics[keepaspectratio]{PA1_template_files/figure-latex/unnamed-chunk-3-1.pdf}}

\begin{Shaded}
\begin{Highlighting}[]
\NormalTok{med}\OtherTok{\textless{}{-}}\FunctionTok{median}\NormalTok{(total\_per\_day}\SpecialCharTok{$}\NormalTok{total\_steps)}
\NormalTok{m}\OtherTok{\textless{}{-}}\FunctionTok{trunc}\NormalTok{(}\FunctionTok{mean}\NormalTok{(total\_per\_day}\SpecialCharTok{$}\NormalTok{total\_steps))}
\end{Highlighting}
\end{Shaded}

The mean of the total number of steps taken per day is 9354 and median
is 10395

\begin{center}\rule{0.5\linewidth}{0.5pt}\end{center}

\subsection{What is the average daily activity
pattern?}\label{what-is-the-average-daily-activity-pattern}

I calculate the average number of steps for each 5-minute interval
across all days to explore daily activity patterns, highlighting the
times of day when activity peaks.

\begin{Shaded}
\begin{Highlighting}[]
\NormalTok{avg\_steps\_interval }\OtherTok{\textless{}{-}}\NormalTok{ activity }\SpecialCharTok{\%\textgreater{}\%}
        \FunctionTok{group\_by}\NormalTok{(interval) }\SpecialCharTok{\%\textgreater{}\%}
        \FunctionTok{summarise}\NormalTok{(}\AttributeTok{mean\_steps=} \FunctionTok{trunc}\NormalTok{(}\FunctionTok{mean}\NormalTok{(steps,}\AttributeTok{na.rm=}\ConstantTok{TRUE}\NormalTok{)))}

\FunctionTok{plot}\NormalTok{(avg\_steps\_interval}\SpecialCharTok{$}\NormalTok{interval,avg\_steps\_interval}\SpecialCharTok{$}\NormalTok{mean\_steps,}
     \AttributeTok{type =} \StringTok{"l"}\NormalTok{,                              }
     \AttributeTok{xlab =} \StringTok{"5{-}minute Interval"}\NormalTok{,}
     \AttributeTok{ylab =} \StringTok{"Average Number of Steps"}\NormalTok{,}
     \AttributeTok{main =} \StringTok{"Average Daily Activity Pattern"}\NormalTok{)}
\end{Highlighting}
\end{Shaded}

\pandocbounded{\includegraphics[keepaspectratio]{PA1_template_files/figure-latex/unnamed-chunk-4-1.pdf}}

\begin{Shaded}
\begin{Highlighting}[]
\NormalTok{max\_interval}\OtherTok{\textless{}{-}}\NormalTok{avg\_steps\_interval}\SpecialCharTok{$}\NormalTok{interval[}\FunctionTok{which.max}\NormalTok{(avg\_steps\_interval}\SpecialCharTok{$}\NormalTok{mean\_steps)]}
\end{Highlighting}
\end{Shaded}

The 5-minute interval, which on average across all the days in the data
set contains the maximum number of steps is 835

\begin{center}\rule{0.5\linewidth}{0.5pt}\end{center}

\subsection{Imputing missing values}\label{imputing-missing-values}

\begin{Shaded}
\begin{Highlighting}[]
\NormalTok{sum\_NA}\OtherTok{\textless{}{-}}\FunctionTok{sum}\NormalTok{(}\FunctionTok{is.na}\NormalTok{(activity}\SpecialCharTok{$}\NormalTok{steps))}
\end{Highlighting}
\end{Shaded}

Since the total number of missing values in the dataset is 2304, I
address them by filling each missing step entry with the average count
for the corresponding 5-minute interval before proceeding with further
analysis.

\begin{Shaded}
\begin{Highlighting}[]
\NormalTok{activity\_filled}\OtherTok{\textless{}{-}}\NormalTok{activity }\SpecialCharTok{\%\textgreater{}\%}
    \FunctionTok{left\_join}\NormalTok{(avg\_steps\_interval, }\AttributeTok{by =} \StringTok{"interval"}\NormalTok{) }\SpecialCharTok{\%\textgreater{}\%}
    \FunctionTok{mutate}\NormalTok{(}\AttributeTok{steps=}\FunctionTok{ifelse}\NormalTok{(}\FunctionTok{is.na}\NormalTok{(steps), mean\_steps,steps)) }\SpecialCharTok{\%\textgreater{}\%}
    \FunctionTok{select}\NormalTok{(steps,date,interval)}
\NormalTok{total\_per\_day\_filled}\OtherTok{\textless{}{-}}\NormalTok{ activity\_filled }\SpecialCharTok{\%\textgreater{}\%}
    \FunctionTok{group\_by}\NormalTok{(date) }\SpecialCharTok{\%\textgreater{}\%}
    \FunctionTok{summarise}\NormalTok{(}\AttributeTok{total\_steps=}\FunctionTok{sum}\NormalTok{(steps,}\AttributeTok{na.rm =} \ConstantTok{TRUE}\NormalTok{))}
\FunctionTok{hist}\NormalTok{(total\_per\_day\_filled}\SpecialCharTok{$}\NormalTok{total\_steps,}
     \AttributeTok{main =} \StringTok{"Total Steps per Day"}\NormalTok{,}
     \AttributeTok{xlab =} \StringTok{"Total Steps"}\NormalTok{,}
     \AttributeTok{ylab =} \StringTok{"Number of Days"}\NormalTok{,}
     \AttributeTok{col =} \StringTok{"blue"}\NormalTok{,}
     \AttributeTok{breaks =} \DecValTok{20}\NormalTok{)}
\end{Highlighting}
\end{Shaded}

\pandocbounded{\includegraphics[keepaspectratio]{PA1_template_files/figure-latex/unnamed-chunk-6-1.pdf}}

\begin{Shaded}
\begin{Highlighting}[]
\NormalTok{med\_filled}\OtherTok{\textless{}{-}}\FunctionTok{median}\NormalTok{(total\_per\_day\_filled}\SpecialCharTok{$}\NormalTok{total\_steps)}
\NormalTok{m\_filled}\OtherTok{\textless{}{-}}\FunctionTok{trunc}\NormalTok{(}\FunctionTok{mean}\NormalTok{(total\_per\_day\_filled}\SpecialCharTok{$}\NormalTok{total\_steps))}
\end{Highlighting}
\end{Shaded}

After imputing missing values, the new mean of the total number of steps
per day is \ensuremath{1.0749\times 10^{4}} (previously 9354), and the
new median is \ensuremath{1.0641\times 10^{4}} (previously 10395).
\textbf{Filling NAs with interval averages isn't fully rigorous, but
it's sufficient for the purpose of this simple analysis.}

\begin{center}\rule{0.5\linewidth}{0.5pt}\end{center}

\subsection{Are there differences in activity patterns between weekdays
and
weekends?}\label{are-there-differences-in-activity-patterns-between-weekdays-and-weekends}

Finally, I examine whether activity patterns differ between weekdays and
weekends by categorizing each day accordingly and calculating the
average steps for each 5-minute interval in both groups.

\begin{Shaded}
\begin{Highlighting}[]
\NormalTok{activity\_filled}\SpecialCharTok{$}\NormalTok{date}\OtherTok{\textless{}{-}}\FunctionTok{as.Date}\NormalTok{(activity\_filled}\SpecialCharTok{$}\NormalTok{date)}
\NormalTok{activity\_filled}\SpecialCharTok{$}\NormalTok{week}\OtherTok{\textless{}{-}} \FunctionTok{ifelse}\NormalTok{(}\FunctionTok{weekdays}\NormalTok{(activity\_filled}\SpecialCharTok{$}\NormalTok{date) }\SpecialCharTok{\%in\%} \FunctionTok{c}\NormalTok{(}\StringTok{"Monday"}\NormalTok{,}\StringTok{"Tuesday"}\NormalTok{,}\StringTok{"Wednesday"}\NormalTok{,}\StringTok{"Thursday"}\NormalTok{,}\StringTok{"Friday"}\NormalTok{),}\StringTok{"weekday"}\NormalTok{,}\StringTok{"weekend"}\NormalTok{)}
\NormalTok{avg\_week}\OtherTok{\textless{}{-}}\NormalTok{ activity\_filled }\SpecialCharTok{\%\textgreater{}\%}
    \FunctionTok{group\_by}\NormalTok{(interval,week) }\SpecialCharTok{\%\textgreater{}\%}
    \FunctionTok{summarise}\NormalTok{(}\AttributeTok{mean\_steps=}\FunctionTok{trunc}\NormalTok{(}\FunctionTok{mean}\NormalTok{(steps,}\AttributeTok{na.rm=}\ConstantTok{TRUE}\NormalTok{)), }\AttributeTok{.groups =} \StringTok{"drop"}\NormalTok{)}
\NormalTok{avg\_weekday}\OtherTok{\textless{}{-}}\NormalTok{ avg\_week }\SpecialCharTok{\%\textgreater{}\%} 
    \FunctionTok{filter}\NormalTok{(week}\SpecialCharTok{==}\StringTok{"weekday"}\NormalTok{)}
\NormalTok{avg\_weekend}\OtherTok{\textless{}{-}}\NormalTok{ avg\_week }\SpecialCharTok{\%\textgreater{}\%} 
    \FunctionTok{filter}\NormalTok{(week}\SpecialCharTok{==}\StringTok{"weekend"}\NormalTok{)}
\FunctionTok{par}\NormalTok{(}\AttributeTok{mfrow =} \FunctionTok{c}\NormalTok{(}\DecValTok{2}\NormalTok{,}\DecValTok{1}\NormalTok{), }\AttributeTok{mar =} \FunctionTok{c}\NormalTok{(}\DecValTok{4}\NormalTok{,}\DecValTok{4}\NormalTok{,}\DecValTok{2}\NormalTok{,}\DecValTok{1}\NormalTok{), }\AttributeTok{oma =} \FunctionTok{c}\NormalTok{(}\DecValTok{0}\NormalTok{,}\DecValTok{0}\NormalTok{,}\DecValTok{4}\NormalTok{,}\DecValTok{0}\NormalTok{))}
\FunctionTok{plot}\NormalTok{(avg\_weekday}\SpecialCharTok{$}\NormalTok{interval,avg\_weekday}\SpecialCharTok{$}\NormalTok{mean\_steps,}
     \AttributeTok{type =} \StringTok{"l"}\NormalTok{,                              }
     \AttributeTok{xlab =} \StringTok{"5{-}minute Interval"}\NormalTok{,}
     \AttributeTok{ylab =} \StringTok{"Average Number of Steps"}\NormalTok{,}
     \AttributeTok{main =} \StringTok{"Weekday"}\NormalTok{)}
\FunctionTok{plot}\NormalTok{(avg\_weekend}\SpecialCharTok{$}\NormalTok{interval,avg\_weekend}\SpecialCharTok{$}\NormalTok{mean\_steps,}
     \AttributeTok{type =} \StringTok{"l"}\NormalTok{,                              }
     \AttributeTok{xlab =} \StringTok{"5{-}minute Interval"}\NormalTok{,}
     \AttributeTok{ylab =} \StringTok{"Average Number of Steps"}\NormalTok{,}
     \AttributeTok{main =} \StringTok{"Weekend"}\NormalTok{)}
\FunctionTok{mtext}\NormalTok{(}\StringTok{"Average Steps per 5{-}Minute Interval: Weekdays vs Weekends"}\NormalTok{,}
      \AttributeTok{side =} \DecValTok{3}\NormalTok{, }\AttributeTok{outer =} \ConstantTok{TRUE}\NormalTok{, }\AttributeTok{line =}\DecValTok{1}\NormalTok{, }\AttributeTok{cex =} \FloatTok{1.5}\NormalTok{)}
\end{Highlighting}
\end{Shaded}

\pandocbounded{\includegraphics[keepaspectratio]{PA1_template_files/figure-latex/panel-plot-1.pdf}}

\begin{center}\rule{0.5\linewidth}{0.5pt}\end{center}

\subsubsection{Thank you for reviewing this
analysis.}\label{thank-you-for-reviewing-this-analysis.}

\emph{Apostolos Karyofyllis}

\end{document}
